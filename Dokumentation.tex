%%This is a very basic article template.
%%There is just one section and two subsections.
\documentclass [a4paper]{article}
\usepackage{german}
\usepackage [utf8]{inputenc}
\usepackage{longtable}
\usepackage{fancyhdr}
\newcommand{\quotes}[1]{``#1''}

\begin{document}
\pagestyle{fancy} %eigener Seitenstil
\fancyhf{} %alle Kopf- und Fußzeilenfelder bereinigen
\fancyhead[L]{Web-basierte Anwendungen || Abgabe 2} %Kopfzeile links
\fancyhead[C]{Web-basierte Anwendungen || Abgabe 2: DVD-Shop Java} %zentrierte
% Kopfzeile
\fancyhead[R]{} %Kopfzeile rechts
\renewcommand{\headrulewidth}{0.4pt} %obere Trennlinie
\fancyfoot[L]{Rebecca Brunclik, Nadine Kreisel, Simon Letychevskyy, Dimitri
Schneider }
\fancyfoot[R]{\thepage}
\renewcommand{\footrulewidth}{0.4pt} %untere Trennlinie
\title{Dokumentation Web-basierte Anwendungen Abgabe 2}
\author{Rebecca Brunclik, Nadine Kreisel, Simon Letychevskyy, Dimitri Schneider }
\maketitle{}
\thispagestyle{empty}
\noindent
\lhead{}
\lfoot{Rebecca Brunclik, Nadine Kreisel, Simon Letychevskyy, Dimitri Schneider }
\newpage
\section{Konfiguration}
Um diese Web-Anwendung verwenden zu können, muss die Datei \-\textbf{Dellstore.war} 
in den gewünschten Webserver integriert werden. Diese Version ist optimiert für den 
\textbf{Apache Tomcat/8.0.20}, der hierfür den Bereich \texttt{Deploy} in der \texttt{Manager App} 
anbietet.\\
Die Anwendung ist auf die Verwendung von PostgreSQL ausgelegt. Auf Ihrer Datenbank sollte die Dellstore-Datenbank 
eingerichtet sein. Hierzu führen Sie bitte das Skript \texttt{dellstore\_reload.sql} aus.
\\
Um die Anwendung an Ihre Umgebung anpassen zu können, liegt im Ordner
\texttt{META-INF} die Konfigurationsdatei \texttt{persistence.xml} bereit. Hier ersetzen Sie bitte
die Werte der \textbf{Properties} \texttt{javax.persistence.jdbc.url}, \texttt{javax.persis\-tence.jdbc.user} 
und \texttt{javax.persistence.jdbc.password} durch Ihren eigenen Datenbank-Host und -Namen, 
den Benutzernamen und das Passwort.\\


\section{Übersicht}
%hier die Seitenübersicht:
%Startseite verlinkt standardmäßig auf die Kunden. Was findet man auf den
% einzelnen Seiten , wie kommt man wo hin etc.

\textbf{Navigation} \\
%was wird wann angezeigt, welche Möglichkeiten hat man
Die Startseite begrüßt den Anwender als Gast. Sie fordert ihn dazu auf, sich einzuloggen und
enthält die dazu benötigte Maske, einen \texttt{Anmelden}- und
einen \texttt{Zurück}-Button. Nach dem Einloggen erscheint eine Willkommensmeldung und die Navigationsleiste wird um den Warenkorb erweitert. Bei falscher
Eingabe wird eine automatisch generierte Warnung oder (bei falschem Passwortformat) Fehlermeldung ausgegeben. \texttt{Zurück} verlinkt auf die Startseite, sodass Eingaben und angezeigte Meldungen gelöscht werden. 
Durchgängig auf allen Seiten wird die Menüleiste angezeigt, die auf die \textbf{Kundenübersicht}, \textbf{Produktübersicht}, 
\textbf{Bestellungsübersicht} sowie im eingeloggten Zustand auf den \textbf{Warenkorb} verlinkt und einen Link für
\textbf{LogIn} bzw. \textbf{LogOut} bereitstellt.
\\


\noindent\textbf{Kundenübersicht}\\
Hier werden jeweils zehn Kunden der gesamten Kundenliste angezeigt. Zur Navigation durch die
Liste kann mit den Buttons \texttt{Vor} und \texttt{Zurück} geblättert oder mit
\texttt{Start}/\texttt{Ende} direkt auf die erste bzw. letzte Seite gesprungen werden.
Durch einen Klick auf die Kopfzeile der jeweiligen Spalte werden die Einträge
nach dieser sortiert. Durch erneutes Anklicken kann die Sortierung der Einträge
umgekehrt werden. Für die Suche in den Kundendaten wird über der Liste ein Suchformular bereitgestellt.\\
\noindent Durch Klicken auf einen Benutzernamen gelangt man zu den zugehörigen
\textbf{Kundendetails}.
\\




\noindent \textbf{Kundendetails editieren}\\
%klick auf editieren gibt die Textfelder frei
Die Textfelder, in denen die Kundendaten angezeigt werden, sind editierbar,
sobald der Button \texttt{Bearbeiten} aktiviert wurde.
%abbrechen 
Durch \texttt{Zurück} kann die Bearbeitung abgebrochen werden. In diesem Fall 
gelangt der Anwender zurück zur Übersicht und es
werden keine Änderungen an die Datenbank übermittelt.
%daten Speichern speichert in DB (transaktional gesichert) und gibt Rückmeldung
Dies geschieht nur, wenn der Button \texttt{Speichern} betätigt wird.\\
% ob gefunzt hat oder nicht

\newpage\noindent\textbf{LogIn / LogOut}\\
%Wie meldet man sich an, welche Möglichkeiten hat man
Hat sich der Anwender über die \texttt{LogIn}-Maske eingeloggt, wird in der
Menüleiste zusätzlich ein Link zu seinem \texttt{Warenkorb} angezeigt.\\
%Produktdetails sind verfügbar inkl. Button in den Warenkorb
Zudem werden die \texttt{Produktnummern} klickbar.\\
Durch einen Klick auf \texttt{LogOut} wird der Nutzer direkt ausgeloggt und 
auf die Startseite verlinkt. Auch ist der \texttt{Warenkorb} dann nicht mehr Bestandteil der Menüleiste.
\\

\noindent\textbf{Warenkorb}\\
Durch Klicken einer \texttt{Produktnummer} gelangt man zu den Produktdetails.
Hier hat man die Möglichkeit, das gewählte Produkt durch Angabe der gewünschten
Menge im Textfeld in den Warenkorb zu legen.
 Positionen im Warenkorb können mit dem Button \texttt{x} gelöscht
werden.
Mit dem Button \texttt{Bestellen} wird der Bestellvorgang abgeschlossen.
Der Button \texttt{Weiter shoppen} führt zurück zur Produktübersicht.\\
Sind beim Bestellen nicht mehr genug Exemplare eines Artikels verfügbar, 
werden diese nicht bestellt und dem Anwender wird der Hinweis darauf angezeigt.
 Loggt sich der Benutzer aus, wird der Warenkorb komplett geleert.
\\

\section{Aufbau}
\textbf{Java Resources}
\begin{longtable}{r | p{6.5cm}}
	\textbf{package controller} & Hauptlogik mit Backing- und Managed\-Beans\\\\ 
	\textbf{package dellstore} & bereitgestellte Persistenzschicht mit \-Entitätsklassen und Datenbank-Controller	\\\\
	\textbf{package persist} & eigene Erweiterung der vorhandenen Persistenz\-schicht \\\\
	\textbf{package util} & Werkzeugklassen \\ &(für Session, Pager, Validator etc.) \\
\end{longtable}
\noindent\textbf{WebContent}
\begin{longtable}{r | p{5.5cm}}
	\textbf{resources/} & CSS-Stylesheet \\\\
	\textbf{WEB-INF/templates/} & xhtml-Templates für Basiselemente, die in die dargestellten Seiten eingebunden werden \\\\
	\textbf{.xhtml} & xhtml-Templates Rendern der gebundenen Daten der ManagedBeans \\
	
	%Views
	%evtl noch aufräumen / Ordnerstruktur

\end{longtable}

\end{document}
